\documentclass[10pt]{beamer}

\usepackage[utf8]{inputenc}
\usepackage[T1]{fontenc}
\usepackage[french]{babel}
\usepackage{fancyhdr}
\usepackage{amsmath}
\usepackage{amssymb}
\usepackage{graphicx}

\usetheme{Warsaw}
\useoutertheme{sidebar}

\title{TD2-Corps finis}
\author{Thierno Mamoudou SABALY}
\institute{Master 1 - Transmission des Données et Sécurité de l'Information\\Option Mathématiques-Cryptographie}
\date{04 janvier 2022}

\AtBeginSection[]
{
  \begin{frame}
  \frametitle{Sommaire}
  \tableofcontents[currentsection, hideothersubsections, pausesubsections]
  \end{frame} 
}

\begin{document}
    \begin{frame}
        \frametitle{\textcolor{white}{Université Cheikh Anta Diop de Dakar}}
        \titlepage
    \end{frame}

    \section{Exercice 01}
	\begin{frame}{Exercice 01}{Question 1}
        \begin{alertblock}{Question 1}
            Montrer que $X^2 + X + 1$ est le seul polynôme unitaire irréductible de dégré 2 sur $\mathbb{F}_2$
        \end{alertblock}
        \begin{exampleblock}{Solution}
            Les autres polynômes unitaires de dégré 2 sur $\mathbb{F}_2$ sont : $X^2 + X$, $X^2 + 1$ et $X^2$.\pause
            Pour chacun d'eux, nous avons bien une factorisation. En effet : 
            \begin{itemize}
                \item $X^2 + X = X(X + 1)$
                \item $X^2 + 1 = (X + 1)(X+1)$
                \item $X^2 = X \times X$
            \end{itemize} \pause
            Alors il ne reste plus qu'à vérifier que $f(X) = X^2 + X + 1$ est irreductible sur $\mathbb{F}_2$, pour cela par l'absurde, supposons que f soit réductible.
        \end{exampleblock}
	\end{frame}
    \begin{frame}{Exercice 01}{Question 01}
        \begin{exampleblock}{}
            Alors $f(X) = (X + a)(X + b); \text{ } a,b \in \mathbb{F}_2$. Donc 0 ou 1 est solution de $f(X) = 0$ or on a
            $f(0) = f(1) = 1$ ce qui est contradictoire. Donc $f$ est irréductible.
        \end{exampleblock}
    \end{frame}

    \begin{frame}{Exercice 01}{Question 02}
        \begin{alertblock}{Question 02}
            Montrer que le polynôme $X^4 + 1$ est irréductible sur $\mathbb{Q}$.
        \end{alertblock}
        \begin{exampleblock}{Solution}
            Soit $P(X) = X^4 + 1$. Par l'absurbe supposons que $P$ est réductible sur $\mathbb{Q}$ alors $P(X) = Q(X)T(X)$ avec $Q, T \in \mathbb{Q}[X]$. On a soit 
             $deg(Q) = 1 \text{ et } deg(T) = 3$ ou $deg(Q) = 2 \text{ et } deg(T) = 2$. \pause
             \begin{enumerate}
                 \item $deg(Q) = 1 \implies \exists \frac{a}{b} \in \mathbb{Q} / P(\frac{a}{b}) = 0 \implies a^4 = - b^4 < 0$ (impossible) \pause
                 \item $deg(Q) = 2$ alors on peut écrire $Q(X) = X^2 + \alpha X + 1$ et $T(X) = X^2 + \beta X + 1$ avec $\alpha, \beta \in \mathbb{Q}$
                        $Q(X)T(X) = X^4 + (\alpha + \beta)X^3 + (2 + \alpha\beta)X^2 + (\alpha + \beta)X + 1$. Par identification, on a :  
            \end{enumerate}
        \end{exampleblock}
    \end{frame}

    \begin{frame}{Exercice 01}{Question 02}
        \begin{exampleblock}{}
            $\alpha + \beta = 0$ et $2 + \alpha\beta = 0 \implies 2 - \alpha^2 = 0 \implies \alpha^2 = 2$. Cette dernière équation n'a pas de solution dans 
            $\mathbb{Q}$.\\ \pause
            
            Alors aucune décomposition de $X^4 + 1$ n'est possible dans $\mathbb{Q}$, d'où $X^4 + 1$ est irréductible dans $\mathbb{Q}$.
        \end{exampleblock}
    \end{frame}

    \begin{frame}{exercice 01}{Question 03}
        \begin{exampleblock}{}
            \begin{itemize}
                \item [(ii)] Si $X^4 + 1 = (X^2 + \alpha X + a)(X^2 + \beta X + a^{-1})$ avec $\alpha, a \in \maathbb{F}_{p}$ alors on a
                $X^4 + 1  = X^4 + (\alpha + \beta)X^3 + (a + a^{-1} + \alpha\beta)X^2 + (a^{-1}\alpha + a\beta)X + 1$;\pause 
                Et donc par identification, 
                \[
                    \left\{
                    \begin{array}{}
                        \alpha + \beta = 0 &
                        a + a^{-1} + \alpha\beta = 0 & 
                        a^{-1}\alpha + a\beta = 0
                    \end{array}
                    \right.
                \]
            \end{itemize}
             
        \end{exampleblock}
    \end{frame}

    \begin{frame}{exercice 01}{Question 03}
        \begin{exampleblock}{}
            On a alors,
            \[
                \left\{
                \begin{array}{}
                    \beta = -\alpha &
                    \alpha^2 = a + a^{-1} & 
                    \alpha(a^{-1} - a) = 0
                \end{array}
                \right.
            \] \pause
            $\implies \alpha = 0 \text{ ou } a = a^{-1}$. Traitons séparément ces 2 cas.\\ \pause
            1. $\alpha = 0 \implies a^{2} = -1 \implies -1 \text{ est un résidu quadratique }$. Par la formule de Legendre, on a $(\frac{-1}{p}) = 1$
            or $(\frac{-1}{p}) = (-1)^\frac{p-1}{2} = 1 \Leftrightarrow p = 8k + 5$.\\ \pause
            \textcolor{blue}{Si $p=8k+5, X^4 + 1 = (X^2 + a)(X^2 - a) \text{ avec } a^2 = -1$}
            
        \end{exampleblock}
    \end{frame}

    \begin{frame}
        \begin{exampleblock}{}
            2. $a = a^{-1} \implies \alpha^2 = 2a$. \pause D'autre part $a = a^{-1} \implies a = 1 \text{ ou } a = p-1 = -1$
            \pause
            \begin{itemize}
                \item Si $p=8k + 3$ alors par la formule de Legendre, $(\frac{2}{p}) = (-1)^{\frac{p^2 - 1}{8}} = -1 \implies 2$ n'est pas résidu quadratique. On choisi $a$ tel que $2a$ soit un résidu quadratique, donc
                $(\frac{2a}{p}) = (\frac{2}{p})(\frac{a}{p}) = -1\times(\frac{a}{p}) \implies (\frac{a}{p}) = -1 \implies a=-1$.\\
                \pause \textcolor{blue}{Si $p=8k+3, X^4 + 1 = (X^2 + \alpha X -1)(X^2 - \alpha X - 1)$ avec $\alpha^2 = -2$}.
                \pause
                \item Si $p=8k + 7$ alors par la formule de Legendre, 2 est un résidu quadratique, il suffit alors de prendre $a=1$. Et on a \\
                \pause \textcolor{blue}{Si $p=8k+7, X^4 + 1 = (X^2 + \alpha X + 1)(X^2 - \alpha X + 1)$ avec $\alpha^2 = 2$ }.
            \end{itemize} \pause

            D'où pour tout nombre premier $p, X^4 + 1$ est réductible sur $\mathbb{F}_{p}$.
        \end{exampleblock}
    \end{frame}

    \begin{frame}{exercice 01}{Question 04}
        \begin{alertblock}{Question 04}
            Déterminer tous les polynômes irréductibles unitaires de degré 3 sur $\mathbb{F}_2$.
        \end{alertblock}
        \begin{exampleblock}{Solution}
            Il est plus aisé de lister les réductibles et ensuite en déduire les irréductibles.\\
            Les réductibles \\ \pause
            \begin{itemize}
                \item $0$ est racine : $X^3,  X(X^2 + X+1) = X^3+X^2+X$ \pause
                \item $0$ et $1$ sont racines : $X(X^2 + 1) = X^3 + X, X(X^2 + X) = X^3 + X^2$ \pause
                \item $1$ est racine : $(X+1)(X^2 + 1) = X^3 + X^2 + X + 1, (X+1)(X^2 + X + 1) = X^3 + 1$ \pause
            \end{itemize}\pause 
            Les irréductibles sont alors : $X^3 + X + 1 \text{ et } X^3 + X^2 + 1$.
        \end{exampleblock}
    \end{frame}

    \begin{frame}{Exercice 01}{Question 05}
        \begin{alertblock}{Question 05}
            Déterminer tous les polynômes irréductibles unitaires de degré 2 sur $\mathbb{F}_3$.
        \end{alertblock}
        \begin{exampleblock}{Solution}
            Un polynôme réductible de degré 2 et unitaire s'écrit $(X+a)(X+b) = X^2 + (a+b)X + ab; a,b \in \mathbb{F}_3$
            Ce sont alors : \pause
            \begin{itemize}
                \item $a=0, b=0 \implies X^2$ \pause 
                \item $a=0, b=1 \text{ ou } a=1, b=0 \implies X^2 + X$ \pause
                \item $a=0, b=2 \text{ ou } a=2, b=0 \implies X^2 + 2X$ \pause
                \item $a=1, b=1 \implies X^2 + 2X + 1$ \pause
                \item $a=1, b=2 \text{ ou } a=2, b=1 \implies X^2 + 2$ \pause
                \item $a=2, b=2 \implies X^2 + X + 1$ \pause
            \end{itemize} \pause
            Ainsi les irréductibles sont : $X^2 + 1 \text{ , } X^2 + X + 2$ et $X^2 + 2X + 2$. 
        \end{exampleblock}
    \end{frame}

    \begin{frame}{Exercice 01}{Question 06}
        \begin{alertblock}{Question 06}
            Les polynômes $A(X) = X^3 + X + 1 \text{ et } B(X) = X^3 + X^2 + 1$, sont-ils irréductibles sur $\mathbb{F}_2$.
        \end{alertblock}
        \begin{exampleblock}{Solution}
            $A(X)$ et $B(X)$ sont irréductibles sur $\mathbb{F}_2$. cf. Question 04.
        \end{exampleblock}
    \end{frame}

    \begin{frame}{Exercice 01}{Question 07}
        \begin{alertblock}{Question 07}
            Lister les éléments de $\mathbb{F}_2[X]/(A)$. En déduire ses tables d'addition et de multiplication correspondantes.
        \end{alertblock}
        \begin{exampleblock}{Solution}
            \begin{itemize}
                Soit $\alpha = X mod(A(x))$, on a : $\alpha^3 + \alpha + 1 = 0 \implies \alpha^3 = \alpha + 1$; 
                On a : \pause
                \begin{tabular}{c|c|c|c}
                    Exponentielle & polynomiale & binaire & décimal \\
                    \hline
                    $\alpha$ & $\alpha$ & 010 & 2 \\
                    \hline
                    $\alpha^2$ & $\alpha^2$ & 100 & 4 \\
                    \hline
                    $\alpha^3$ & $\alpha + 1 $ & 011 & 3 \\
                    \hline
                    $\alpha^4$ & $\alpha^2 + \alpha $ & 110 & 6 \\
                    \hline
                    $\alpha^5$ & $\alpha^2 + \alpha + 1 $ & 111 & 7 \\
                    \hline
                    $\alpha^6$ & $\alpha^2 + 1 $ & 101 & 5 \\
                    \hline
                    $\alpha^7$ & $1$ & 001 & 1 \\
                    \hline
                \end{tabular}
            \end{itemize}
        \end{exampleblock}
    \end{frame}
    

    \section{Exercice 02}
    \section{Exercice 03}
 
    \section{Exercice 04}
  

    \section{Exercice 05}
    \section{Exercice 06}
    \section{Exercice 07}


        \begin{frame}{Exercice 07}{Question 02}
            \begin{alertblock}{Question 02}
                Déterminer le cardinale de $\mathbb{F}$.
            \end{alertblock}
            \begin{exampleblock}{Solution}
                $\mathbb{F} = \{ a_0 + a_1\alpha + ... + a_{m-1}\alpha^{m-1}, a_i \in \mathbb{F}\} \backsimeq \{(a_i)_{0\le i \le m-1}, a_i \in \mathbb{F}_p\}$. On peut voir cet ensemble comme l'ensemble des m-uplets de $\mathbb{F}_p$.
                Donc son cardinale c'est le nombre d'uplets possibles, c'est-à-dire $p^m$.\\
                $card(\mathbb{F}) = p^m$.
            \end{exampleblock}
        \end{frame}

        \begin{frame}{Exercice 07}{Question 03}
            \begin{alertblock}{Question 03}
                Montrer que $\mathbb{F}$, muni de l'addition $+$ des polynômes et de $\cdot$ multiplication de polynômes modulo $g$, est un corps.
            \end{alertblock}
            \begin{exampleblock}{Solution}
                \begin{itemize}
                    \item La somme de deux polynômes de degré inférieur ou égale à m-1 à coefficients dans $\mathbb{F}_p$ est un polynôme de degré inférieur ou égale à m-1 et à coefficients dans $\mathbb{F}_p$ (car $\mathbb{F}_p$ est un corps). Stabilité par somme \pause
                    \item Le produit module $g$ de deux polynômes de degré inférieur ou égale à m-1 est aussi un polynôme de degré inférieur ou égale à m-1 et à coefficients dans $\mathbb{F}_p$. Stabilité par produit \pause
                    \item Soit $P  \in \mathbb{F}_p[X]$, alors comme $g$ irréductible alors soit $pgcd(P,g) = g$ ou $pgcd(P,g) = 1 \implies P = 0$ ou $P$ inversible. Or tout élement $P$ non nul de $\mathbb{F}$ est de degré inférieur ou égale à $deg(g) \implies pgcd(P, g) = 1 \implies P$ inversible. 
                \end{itemize} 
            \end{exampleblock}
        \end{frame}

        \begin{frame}{Exercice 07}{Question 03}
            \begin{example}{}
                Donc $(\mathbb{F}, +, \cdot)$ est un corps.
            \end{example}
        \end{frame}

        \begin{frame}{Exercice 07}{Question 04}
            \begin{alertblock}{Question 04}
                Montrer que $\mathbb{F}$ est isomorphe à $\mathbb{F}_p[X]/(g)$.
            \end{alertblock}
            \begin{alertblock}{Solution}
                Soit l'application : $\phi : \mathbb{F}_p[X] \to \mathbb{F}; P \mapsto P(\alpha)$. \pause
                \begin{itemize}
                    \item Surjection : soit $Y \in \mathbb{F} \implies Y = \sum_{k=0}^{m-1} a_k\alpha^k = P(\alpha)$ avec $P(X) = \sum_{k=0}^{m-1} a_kX^k \in \mathbb{F}_p[X]$. \pause
                    \item $ker\phi = \{ P \in \mathbb{F}_p[X] / P(\alpha) = 0\} = (g)$.\pause En effet, $\forall P \in (g), P(X) = g(X)Q(x) \implies P(\alpha) = g(\alpha)Q(\alpha) = 0$ 
                        \\ \pause  Et $\forall P \in \mathbb{F}_p[X], P(\alpha) = 0$
                         alors $deg(P) > deg(g)$, on a $P(X) = g(X)Q(X) + R(X), deg(R) = 0$ ou $deg(R) < deg(g)$.\pause Alors $P(\alpha) = R(\alpha) = 0 \implies deg(R) = 0$ par minimalité de $g$. \pause D'où $R(X) = 0 \implies P(X) = g(X)Q(X) \in (g)$. 
                \end{itemize} \pause
                D'après le premier théorème d'isomorphisme, on a : $\mathbb{F}_p[X]/(g) \backsimeq \mathbb{F}$.
            \end{alertblock}
        \end{frame}

        \begin{frame}{Exercice 07}{Question 05}
            \begin{alertblock}{Question 05}
                Donner de manière explicite $\mathbb{F}_{256}$.
            \end{alertblock}
            \begin{exampleblock}{Solution}
                De ce qui précède, $\mathbb{F}_{256} = \mathbb{F}_{2^8}$ c'est l'ensemble des polynômes de degré inférieur ou égale à 7 et à coefficients dans $\mathbb{F}_2$. \\ \pause
                
                D'où $\mathbb{F}_{256} = \{\sum_{k=0}^{7} a_kx^k; a_k \in\{0,1\}\}$.
            \end{exampleblock}
        \end{frame}

        \begin{frame}{Exercice 07}{question 06}
            \begin{alertblock}{Question 06}
                Expliciter un isomorphisme entre $\mathbb{F}_{81}$ et $\mathbb{F}_9[X]/(P)$, avec $P$ à déterminer.
            \end{alertblock}
            \begin{block}{Solution !!!}
                Le polynôme $g(X) = X^4 + 1$ est unitaire et irréductible sur $\mathbb{F}_9$. De ce qui précède, $\mathbb{F}_{81}  \backsimeq \mathbb{F}_3[X]/(g)$.\pause
                Et on a $ \tilde{\phi} : \mathbb{F}_3[X]/(g) \to \mathbb{F}_{81}; \tilde{f} \mapsto \tilde{f}(\alpha)$. \pause \\
                avec $\alpha = X mod(g(X)) \implies g(\alpha) = 0 \implies  \alpha^4 = -1$. \\ \pause 
                Soit $P(X) = X^2 + X + 1$, P est irréductible sur $\mathbb{F}_9$ donc $\mathbb{F}_9[X]/(P)$ est un corps. \\ \pause
                Les éléments de $\mathbb{F}_9[X]/(P)$ sont des polynômes de la forme $aX + b$ avec $a,b \in \mathbb{F}_9$. Cherchons une racine de $g$ dans $\mathbb{F}_9[X]/(P)$ \pause \\
                $g(aX + b) = (aX+b)^4 + 1 =  2a^3bX^3 + 6a^2b^2X^2 + 4ab^3X + b^4 - a^4 + 1 = 0 \implies b = 0$ et $ a^4 = 1$. On peut alors prendre $a=1 \implies aX+b = X mod(P(X)) = \beta$.
            \end{block}
        \end{frame}
    \section{Exercice 08}
        \begin{frame}{Exercice 08}{Question 01}
            
        \end{frame}
\end{document}
